\documentclass[a4paper,norsk, 10pt]{article}
\usepackage[utf8]{inputenc}
\usepackage{verbatim}
\usepackage{listings}
\usepackage{graphicx}
\usepackage[norsk]{babel}
\usepackage{a4wide}
\usepackage{color}
\usepackage{amsmath}
\usepackage{float}
\usepackage{amssymb}
\usepackage[dvips]{epsfig}
\usepackage[toc,page]{appendix}
\usepackage[T1]{fontenc}
\usepackage{cite} % [2,3,4] --> [2--4]
\usepackage{shadow}
\usepackage{hyperref}
\usepackage{titling}
\usepackage{marvosym }
\usepackage{subcaption}
\usepackage[noabbrev]{cleveref}
\usepackage{cite}


\setlength{\droptitle}{-10em}   % This is your set screw

\setcounter{tocdepth}{2}

\lstset{language=c++}
\lstset{alsolanguage=[90]Fortran}
\lstset{alsolanguage=Python}
\lstset{basicstyle=\small}
\lstset{backgroundcolor=\color{white}}
\lstset{frame=single}
\lstset{stringstyle=\ttfamily}
\lstset{keywordstyle=\color{red}\bfseries}
\lstset{commentstyle=\itshape\color{blue}}
\lstset{showspaces=false}
\lstset{showstringspaces=false}
\lstset{showtabs=false}
\lstset{breaklines}
\title{FYS3120 Oblig 4}
\author{Daniel Heinesen, daniehei}
\begin{document}
\maketitle

\section*{1)}
\subsection*{a)}

We are going to start by finding the coordinates, we only need to find $y$, since we need it to find the potential energy

$$
y = -\frac{b}{2}\cos\theta
$$

The potential energy is then given by

$$
V = mgy = -mg\frac{b}{2}\cos\theta
$$

There is only rotational energy for the rod, it has a inertia of $I = \frac{1}{3}mb^2$. So its kinetic energy is given as

$$
T = \frac{1}{2}I\omega^2 = \frac{1}{6}mb^2\dot{\theta}^2
$$

Giving us the Lagrangian

$$
L = T -V = \frac{1}{6}mb^2\dot{\theta}^2 +mg\frac{b}{2}\cos\theta
$$

We can now find the e.o.m

$$
\frac{\partial L}{\partial \theta} = -\frac{mgb}{2}\sin\theta
$$

$$
\frac{\partial L}{\partial \dot{\theta}} = \frac{1}{3}mb^2\dot{\theta}, \qquad \frac{d}{dt}\left(\frac{\partial L}{\partial \dot{\theta}} \right) = \frac{1}{3}mb^2\ddot{\theta}
$$

Giving us that

$$
\frac{1}{3}mb^2\ddot{\theta} +\frac{mgb}{2}\sin\theta = 0
$$

or

$$
\ddot{\theta} + \frac{3g}{2b}\sin\theta = 0
$$

\subsection*{b)}
We can see that for $\theta = 0$

$$
\ddot{\theta} = 0
$$


So this is a equilibrium. This is a minimum or maximum of the potential. We need to differentiate the potential twice to see if this is a minimum

$$
\frac{d^2}{d\theta^2} -\frac{mgb}{2}\cos\theta = \frac{mgb}{2}\cos\theta
$$
at $\theta = 0$
$$
\Rightarrow \frac{mgb}{2}\cos(0) = \frac{mgb}{2} > 0
$$
This shows that we have a minimum, and the equilibrium is thus stable.\\

We can now look at the e.o.m for small angles. Then $\sin \theta \approx \theta$, so

$$
\ddot{\theta} = - \frac{3g}{2b}\theta
$$

This is harmonic oscillator with 

$$
\omega^2 = \frac{3g}{2b}
$$

Using that $T = \frac{2\pi}{\omega}$ we find that

$$
T_0 = 2\pi \sqrt{\frac{2b}{3g}}
$$

\subsection*{c)}

Since 

$$
\frac{\partial L}{\partial t} = 0
$$

and 

$$
\frac{dH}{dt} = -\frac{\partial L}{\partial t} = 0
$$

We can see that the Hamiltonian is a constant of motion. Since the constraints are holonomic and time-independent, we know that the Hamiltonian is the same as the total energy of the system. Thus the total energy is conserved.\\

The Hamiltonian is related to the e.o.m by Hamilton's equations

$$
\dot{p} = -\frac{\partial H}{\partial \theta},\qquad \dot{\theta} = \frac{\partial H}{\partial p}
$$

\subsection*{d)}
\textit{This derivation is heavily inspired by a similar derivation for a normal pendulum, found on wikipedia.}\\

We are going to start by looking at the energy of the pendulum. The pendulum has no kinetic energy at $\theta_0$ and potential $mg\cos\theta_0$. At some $\theta$ the kinetic energy has to be the difference in potential energy between these angles. So

$$
\frac{1}{6}mb^2\dot{\theta}^2 = \frac{mgb}{2}(\cos\theta - \cos\theta_0)
$$

giving us that

$$
\dot{\theta} = \frac{d\theta}{dt} = \sqrt{\frac{3g}{b}}\sqrt{(\cos\theta - \cos\theta_0)}
$$

Doing some physicist math we get that

$$
dt = \sqrt{\frac{b}{3g}}\frac{d\theta}{\sqrt{\cos\theta - \cos\theta_0}}
$$

The pendulum needs to fall from $0$ to $\theta_0$, then to $-\theta_0$, and then all the way back to $\theta_0$, meaning that the period is 4 times the time it take to fall from $0$ to $\theta_0$. Meaning that 

$$
T = 4\int dt = 4\sqrt{\frac{b}{3g}}\int_0^{\theta_0} \frac{d\theta}{\sqrt{\cos\theta - \cos\theta_0}}
$$

And we know that $T_0 = 2\pi \sqrt{\frac{2b}{3g}}$, giving us that

$$
T = T_0 \frac{\sqrt{2}}{\pi}\int_0^{\theta_0} \frac{d\theta}{\sqrt{\cos\theta - \cos\theta_0}}
$$

We can now solve this for $\theta_0 = \pi/2$

$$
T = T_0 \frac{\sqrt{2}}{\pi}\int_0^{\pi/2} \frac{d\theta}{\sqrt{\cos\theta}}
$$

From good old Rottmann page 161 eq. 116, we get that this is

$$
T = T_0 \frac{\sqrt{2}}{\pi} \frac{\pi \Gamma(\frac{1}{2})}{\sqrt{2}\Gamma(\frac{3}{4})^2} = T_0\frac{\sqrt{\pi}}{\Gamma(\frac{3}{4})^2} \approx 1.18 \cdot T_0
$$

\section*{2)}
\subsection*{a)}
First we write the coordinates of the ball

$$
x = r\sin\theta \cos(\omega t), \qquad y = r\sin\theta \sin(\omega t), \qquad z = -r\cos \theta
$$

$$
\dot{x} = r\dot{\theta}\cos\theta \cos(\omega t) - r\omega\sin\theta \sin(\omega t) 
$$

$$
\dot{y} = r\dot{\theta}\cos\theta \sin(\omega t) - r\omega\sin\theta \cos(\omega t) 
$$

$$
\dot{z} = r\dot{\theta}\sin \theta
$$

with out writing down the quite long expressions for $\dot{x}$, $\dot{y}$ and $\dot{z}$, we get that

$$
\dot{x}^2 + \dot{y}^2 + \dot{z}^2 = r^2\dot{\theta}^2 + r^2\omega^2\sin^2 \theta
$$

We have the energies 

$$
V = mgy = -mgr\cos\theta
$$
$$
T = \frac{1}{2}m(r^2\dot{\theta}^2 +r^2\omega^2\sin^2\theta)
$$

Giving us the Lagrangian

$$
L = \frac{1}{2}m(r^2\dot{\theta}^2 +r^2\omega^2\sin^2\theta) + mgr\cos\theta
$$

We can rewrite this as

$$
L = \frac{1}{2}mr^2\dot{\theta}^2+ (\frac{1}{2}mr^2\omega^2\sin^2\theta + mgr\cos\theta)
$$

We can therefor say that we have a potential

$$
W(\theta) = -(\frac{1}{2}mr^2\omega^2\sin^2\theta + mgr\cos\theta)
$$

The term 

$$
-\frac{1}{2}mr^2\omega^2\sin^2\theta
$$

Can be said to be the potential energy due to the centripetal force.

\subsection*{b)}

We can find the e.o.m

$$
\frac{\partial L}{\partial \theta} = mr^2\omega^2\sin\theta\cos\theta - mgr\sin\theta =  \frac{1}{2}mr^2\omega^2\sin 2\theta - mgr\sin\theta
$$

$$
\frac{\partial L}{\partial \dot{\theta}} = mr^2\dot{\theta}, \qquad \frac{d}{dt}\left(\frac{\partial L}{\partial \dot{\theta}}\right) = mr^2\ddot{\theta}
$$

Giving us that

$$
mr^2\ddot{\theta} - \frac{1}{2}mr^2\omega^2\sin 2\theta + mgr\sin\theta = 0
$$

We can look at this for small angles. For small angles $\sin \theta \approx \theta$, so

$$
mr^2\ddot{\theta} - mr^2\omega^2\theta +mg\theta = 0
$$

$$
\Rightarrow \ddot{\theta} = -\left(\frac{g}{r} - \omega^2\right)\theta
$$

This is a harmonic oscillation, giving us

$$
\Omega = \sqrt{\frac{g-r\omega^2}{r}}
$$

\subsection*{c)}
We have the potential

$$
W(\theta) = -(\frac{1}{2}mr^2\omega^2\sin^2\theta + mgr\cos\theta)
$$

To find the equilibrium positions we have to differentiate

\begin{equation}
W'(\theta) = -mr^2\omega^2\sin\theta\cos\theta + mgr\sin\theta = -\frac{1}{2} mr^2\omega^2\sin 2\theta + mgr\sin\theta
\label{eq:1der}
\end{equation}

We can see that this is 0 for $\theta = 0$, which indeed is an equilibrium. To find if it is a stable equilibrium, we have to check for which $\omega$ $W''(\theta = 0) > 0$

\begin{equation}
W''(\theta) = -mr^2\omega^2\cos 2\theta + mgr\cos\theta
\label{eq:2der}
\end{equation}

$$
W''(0) = -mr^2\omega^2 +mgr > 0
$$

This gives us the condition that

$$
\omega < \sqrt{\frac{g}{r}} = \omega_{cr}
$$

We are now going to look at $\omega > \omega_{cr}$. We are going to presuppose that $\sin\theta \neq 0$, and can therefor rewrite $W'(\theta)$ \ref{eq:1der} as

$$
-mr^2\omega_{cr}\cos\theta = mgr
$$

$$
\Rightarrow \cos\theta = \frac{g}{r\omega^2}
$$

Giving us the new equilibrium points

$$
\theta_{\pm} = \arccos(\frac{g}{r\omega^2}) + n\pi, \qquad n= 0,1
$$

\subsection*{d)}

When we now are going to check if these points are stable I am going to use a trick which makes it alot easier. I am going to say that 

$$
\omega = a\omega_{cr}
$$

Where $a>1$. This gives us that the equilibrium points can be written as

$$
\theta_{\pm} = \arccos(\frac{1}{a^2}) + n\pi, \qquad n= 0,1
$$

We can now use this to find if $W''(\theta_{\pm}) > 0$. Using \ref{eq:2der}

$$
W''(\arccos(\frac{1}{a^2})) =  -mr^2\omega^2\cos 2(\arccos(\frac{1}{a^2})) + mgr\cos(\arccos(\frac{1}{a^2}))
$$

We can obviously see that $\cos(2(\arccos(x))) = 2x^2 - 1$, so

$$
-mr^2\omega^2\left(\frac{2}{a^2}-1\right) + \frac{mgr}{a^2} > 0
$$

$$
\Rightarrow \frac{g}{a^2} > ra^2\omega_{cr}\left(\frac{2}{a^2}-1\right)
$$

$$
\Rightarrow \frac{1}{a^2} > 2-a^2
$$

$$
\Rightarrow 1> 2a^2 - a^4 = a^2(2-a^2)
$$

Since $\cos(2(\arccos(x)) + n\pi) = 2x^2 - 1$ we get the same inequality for $W''(\arccos(\frac{1}{a^2})+n\pi)$.\\

Looking at the inequality $1 > a^2(2-a^2)$ for $a = 1 \Rightarrow a^2(2-a^2) = 1$, we defined $a>1$ and for $a>1$ the function $a^2(2-a^2)$ is monotonically decreasing function, thus the inequality holds, and $\theta_{\pm}$ are stable equilibriums.


\subsection*{e)}
The Hamiltonian is defined here as

\begin{equation}
H = p\dot{\theta} - L
\label{eq:hamiltonian}
\end{equation}

where

$$
p = \frac{\partial L}{\partial \dot{\theta}} = mr^2\dot{\theta}
$$

We then get a Hamiltonian

$$
H = mr^2\dot{\theta}^2 - L = \frac{p}{mr^2}- L 
$$

$$
= \frac{p}{mr^2} - \frac{p}{2mr^2} - \frac{1}{2}mr^2\omega^2\sin^2\theta - mgr\cos\theta
$$
$$
= \frac{p}{2mr^2} - \frac{1}{2}mr^2\omega^2\sin^2\theta - mgr\cos\theta
$$
\end{document}


