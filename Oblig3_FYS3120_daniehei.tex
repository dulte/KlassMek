\documentclass[a4paper,norsk, 10pt]{article}
\usepackage[utf8]{inputenc}
\usepackage{verbatim}
\usepackage{listings}
\usepackage{graphicx}
\usepackage[norsk]{babel}
\usepackage{a4wide}
\usepackage{color}
\usepackage{amsmath}
\usepackage{float}
\usepackage{amssymb}
\usepackage[dvips]{epsfig}
\usepackage[toc,page]{appendix}
\usepackage[T1]{fontenc}
\usepackage{cite} % [2,3,4] --> [2--4]
\usepackage{shadow}
\usepackage{hyperref}
\usepackage{titling}
\usepackage{marvosym }
\usepackage{subcaption}
\usepackage[noabbrev]{cleveref}
\usepackage{cite}


\setlength{\droptitle}{-10em}   % This is your set screw

\setcounter{tocdepth}{2}

\lstset{language=c++}
\lstset{alsolanguage=[90]Fortran}
\lstset{alsolanguage=Python}
\lstset{basicstyle=\small}
\lstset{backgroundcolor=\color{white}}
\lstset{frame=single}
\lstset{stringstyle=\ttfamily}
\lstset{keywordstyle=\color{red}\bfseries}
\lstset{commentstyle=\itshape\color{blue}}
\lstset{showspaces=false}
\lstset{showstringspaces=false}
\lstset{showtabs=false}
\lstset{breaklines}
\title{FYS3121 Oblig 3}
\author{Daniel Heinesen, daniehei}
\begin{document}
\maketitle

\section*{1)}
\subsection*{a)}

To find the Lagrangian, we need to find the Cartesian coordinates expressed with the general coordinate $s$

\begin{equation}
y = h -s\sin 30^{\circ} = h-\frac{s}{2}, \qquad x = s\cos 30^{\circ} = s\frac{\sqrt{3}}{2}
\end{equation}

We then find that

$$
\dot{y} = -\frac{\dot{s}}{2}, \qquad \dot{x} = \dot{s}\frac{\sqrt{3}}{2}
$$

We can then write the energies

$$
V = mgy = mg(h-\frac{s}{2})
$$
$$
T = \frac{1}{2}mv^2 = \frac{1}{2}m(\frac{\dot{s}^2}{4} + \frac{3}{4}\dot{s}^2) = \frac{1}{2}m\dot{s}^2
$$

We then find that the Lagrangian is

$$
\mathcal{L} = T -V = \frac{1}{2}m\dot{s}^2 -mg(h-\frac{s}{2}) 
$$

\subsection*{b)}
With an accelerated incline, we have to rewrite the coordinates

$$
y = h-\frac{s}{2}, \qquad x = \frac{1}{2}at^2 + s\frac{\sqrt{3}}{2}
$$

$$
\dot{y} = -\frac{\dot{s}}{2}, \qquad \dot{x} = at + \dot{s}\frac{\sqrt{3}}{2}
$$

For the energies we see that

$$
V = mgy = mg(h-\frac{s}{2})
$$

$$
T = \frac{1}{2}mv^2 = \frac{1}{2}m(\frac{\dot{s}^2}{4} + a^2t^2 + \dot{s}at\sqrt{3} + \dot{s}^2\frac{3}{4})
$$

$$
= \frac{1}{2}m()\dot{s}^2 + at(at + \dot{s}\sqrt{3}))
$$

So we get the Lagrangian 

$$
\mathcal{L} = T -V = \frac{1}{2}m(\dot{s}^2 + at(at + \dot{s}\sqrt{3})) - mg(h-\frac{s}{2})
$$

\subsection*{c)}
We can now find the e.o.m. for this system by solving Lagrange's equation. We first find that

$$
\frac{\partial \mathcal{L}}{\partial s} = \frac{mg}{2}
$$

$$
\frac{\partial \mathcal{L}}{\partial \dot{s}} = m\dot{s} +mat\frac{\sqrt{3}}{2}
$$

$$
\frac{d}{dt}\left(\frac{\partial \mathcal{L}}{\partial \dot{s}} \right) = m\ddot{s} + ma\frac{\sqrt{3}}{2}
$$

We can then write the e.o.m

$$
m\ddot{s} + ma\frac{\sqrt{3}}{2} - \frac{mg}{2} = 0
$$

$$
\Rightarrow \ddot{s} = \frac{1}{2}(g-\sqrt{3}a)
$$

We can now integrate this twice with respect to $s$.

$$
s(t) = s_0 +v_0t + \frac{1}{4}(g-\sqrt{3}a)t^2
$$

Knowing that at $t=0$: $r_0 = 0$ and $v_0 = 0$ we finally get that

$$
s(t) = \frac{1}{4}(g-\sqrt{3}a)t^2
$$

\section*{2)}
First we write the coordinates of the ball

$$
x = r\sin\theta \cos(\omega t), \qquad y = r\sin\theta \sin(\omega t), \qquad z = r\cos \theta
$$

$$
\dot{x} = r\dot{\theta}\cos\theta \cos(\omega t) - r\omega\sin\theta \sin(\omega t) 
$$

$$
\dot{y} = r\dot{\theta}\cos\theta \sin(\omega t) - r\omega\sin\theta \cos(\omega t) 
$$

$$
\dot{z} = -r\dot{\theta}\sin \theta
$$

with out writing down the quite long expressions for $\dot{x}$, $\dot{y}$ and $\dot{z}$, we get that

$$
\dot{x}^2 + \dot{y}^2 + \dot{z}^2 = r^2\dot{\theta}^2 + r^2\omega^2\sin^2 \theta
$$

Since there are no gravitation, we get that $V = 0$, and therefor

$$
\mathcal{L} = T = \frac{1}{2}m(r^2\dot{\theta}^2 + r^2\omega^2\sin^2\theta)
$$

We can now find Lagrange's equation

$$
\frac{\partial \mathcal{L}}{\partial \theta} = mr^2\omega^2 \sin\theta \cos\theta
$$

$$
\frac{\partial \mathcal{L}}{\partial \dot{\theta}} = mr^2\dot{\theta}
$$

$$
\frac{d}{dt}\left( \frac{\partial \mathcal{L}}{\partial \dot{\theta}} \right) = mr^2\ddot{\theta}
$$

An we find the e.o.m

$$
mr^2\ddot{\theta} - mr^2\omega^2 \sin\theta \cos\theta = 0
$$

\subsection*{b)}
A particle in a periodic potential $V(\theta)$ would have a Lagrangian

$$
\mathcal{L} = \frac{1}{2}mr^2\dot{\theta}^2 - V(\theta)
$$

We then look at out Lagrangian

$$
\mathcal{L} = \frac{1}{2}m(r^2\dot{\theta}^2 + r^2\omega^2\sin^2\theta)
$$

we then get
$$
\mathcal{L} = \frac{1}{2}mr^2(\dot{\theta}^2 + \omega^2\sin^2\theta) = \frac{1}{2}mr^2\dot{\theta}^2 + \frac{1}{2}mr^2 \omega^2\sin^2\theta
$$

We can see that we now have something that looks like a potential. We can say that the potential of this Lagrangian is

$$
V(\theta) = -\frac{1}{2}mr^2\omega^2\sin^2\theta
$$

This is a periodic potential with 2 stable equilibriums (at $0$ and $\pi$) and 2 unstable equilibriums(at $\pi/2$ and $3\pi/2$). So we can say that the Lagrangian of this system is similar to that of a particle in a periodic potential.

\subsection*{c)}

We can now start with out Lagrangian

$$
\mathcal{L} = \frac{1}{2}mr^2\dot{\theta}^2 + \frac{1}{2}mr^2 \omega^2\sin^2\theta
$$

and find the e.o.m

$$
\frac{\partial \mathcal{L}}{\partial \theta} = mr^2\omega^2\sin\theta\cos\theta
$$

$$
\frac{\partial \mathcal{L}}{\partial \dot{\theta}} = mr^2\dot{\theta}
$$
$$
\frac{d}{dt}\left(\frac{\partial \mathcal{L}}{\partial \dot{\theta}}\right) = mr^2\ddot{\theta}
$$

So the e.o.m is 

$$
mr^2\ddot{\theta} - mr^2\omega^2\sin\theta\cos\theta = 0
$$

We are now going to define $\phi = \theta - \theta_0$, where $\theta_0 = \frac{\pi}{2} + n\pi$. This gives us that $\theta = \theta_0 + \phi$, which we can use in the e.o.m

$$
mr^2\ddot{\phi} - mr^2\omega^2\sin(\phi + \theta_0)\cos(\phi + \theta_0) = 0
$$

We can now rewrite

$$
sin(\phi + \theta_0)\cos(\phi + \theta_0) = (\cos(\phi)\sin(\theta_0) + \cos(\theta_0)\sin(\phi))(\cos(\theta_0)\cos(\phi) - \sin(\phi)\sin(\theta_0))
$$

An using that $\theta = \theta_0 + \phi$

$$
sin(\phi + \theta_0)\cos(\phi + \theta_0) = -\sin(\phi)\cos(\phi) = -\frac{1}{2}\sin(2\phi)
$$

So

$$
mr^2\ddot{\phi} +\frac{1}{2}mr^2\omega^2\sin(2\phi) = 0
$$

From small angles, we know that $\sin\phi \approx \phi$. Using this we get

$$
mr^2\ddot{\phi} +mr^2\omega^2\phi = 0
$$

Which gives us

$$
\ddot{\phi} = \omega^2\phi
$$

This is a harmonic oscillator, with a angular frequency given simply (and tautologically) as

$$
\omega = \omega
$$



\section*{3)}
\subsection*{a)}
Here we start by finding the energies. If we define the plate as being where $V = 0$, we can see that only the lower mass has a potential energy. If the rope has length $L$, we see that

$$
V = mg(-(L-r)) = mg(r-L)
$$

The lower mass have a kinetic energy gives as 

$$
\frac{1}{2}m\dot{r}^2
$$

While the upper mass has kinetic energy from being dragged towards the hole, and from the rotation around the hole. Its kinetic energy is then

$$
\frac{1}{2}m\dot{r}^2 + \frac{1}{2}mr^2\dot{\theta}^2
$$

Giving us a total kinetic energy of

$$
T = 2\cdot\frac{1}{2}m\dot{r}^2 + \frac{1}{2}mr^2\dot{\theta}^2 
$$

And a Lagrangian

$$
\mathcal{L} = \frac{1}{2}m(2\dot{r}^2 +r^2\dot{\theta}^2) - mg(r-L)
$$

We can now find the e.o.m

\begin{equation}
\frac{\partial \mathcal{L}}{\partial \theta} = 0
\label{eq:partialtheta}
\end{equation}


\begin{equation}
\frac{\partial \mathcal{L}}{\partial \dot{\theta}} = mr^2\dot{\theta}
\label{eq:partialthetadot}
\end{equation}

$$
\frac{d}{dt}\left(\frac{\partial \mathcal{L}}{\partial \dot{\theta}}\right) = mr^2\ddot{\theta} + 2mr\dot{r}\dot{\theta}
$$

So the first e.o.m is:

$$
mr^2\ddot{\theta} + 2mr\dot{r}\dot{\theta} = 0
$$


For the second

$$
\frac{\partial \mathcal{L}}{\partial r} = mr\dot{\theta}^2 - mg
$$
$$
\frac{\partial \mathcal{L}}{\partial \dot{r}} = 2m\dot{r}
$$
$$
\frac{d}{dt}\left(\frac{\partial \mathcal{L}}{\partial \dot{r}}\right) = 2m\ddot{r}
$$

So the second e.o.m becomes

$$
2m\ddot{r} - mr\dot{\theta}^2 + mg = 0
$$

\subsection*{b)}

As we saw above from \ref{eq:partialtheta}

$$
\frac{\partial \mathcal{L}}{\partial \theta} = 0
$$

This means that $\theta$ is a cyclic coordinate, and therefor


$$
\frac{\partial \mathcal{L}}{\partial \dot{\theta}} = mr^2\dot{\theta} = \mathrm{constant} = \ell
$$

We can there for rewrite $\dot{\theta}$ as 

$$
\dot{\theta} = \frac{\ell}{mr^2}
$$

If we set this into the second e.o.m we get a description of the system, only dependent on $r$

$$
2m\ddot{r} - \frac{\ell ^2}{mr^3} + mg = 0
$$

\section*{4)}
\subsection*{a)}

For convenience I am going to label the upper mass $m_1$ and the pendulum bob $m_2$. The coordinates then becomes

$$
x_1 = s, \qquad y_1 = 0
$$
$$
\dot{x}_1 = \dot{s}, \qquad \dot{y}_1 = 0
$$

and

$$
x_2 =  s + d\sin \theta, \qquad y_2 = -d\cos \theta
$$

$$
\dot{x}_2 = \dot{s} + d\cos\theta,\qquad \dot{y}_2 = d\sin\theta
$$

And the energies

$$
V = mg(y_1 + y_2) = -mgd\cos\theta
$$
$$
T = \frac{1}{2}m(\dot{x}_1^2 + \dot{y}_1^2 + \dot{x}_2^2 + \dot{y}_2^2)
$$
$$
= \frac{1}{2}m(2\dot{s}^2 +d^2\dot{\theta}^2 + 2\dot{\theta}\dot{s}d\cos\theta) 
$$

This gives us

$$
\mathcal{L} = T -V = \frac{1}{2}m(2\dot{s}^2 +d^2\dot{\theta}^2 + 2\dot{\theta}\dot{s}d\cos\theta) +mgd\cos\theta
$$

\subsection*{b)}
If we differentiate $\mathcal{L}$ with respect to $s$

$$
\frac{\partial \mathcal{L}}{\partial s} = 0
$$

So $s$ is a cyclic coordinate, meaning that the Lagrangian does not depend on $s$. It can also be shown that since $s$ is cyclic, there is a conserved quantity in this system, namely

$$
\frac{\partial \mathcal{L}}{\partial \dot{s}} = 2m\dot{s} + md\dot{\theta}\cos\theta 
$$

Since this quantity is conserved, it is constant

$$
2m\dot{s} + md\dot{\theta}\cos\theta = \ell
$$

We can therefor rewrite this as 

$$
\dot{s} =\frac{1}{2}(\frac{\ell}{m} -d\dot{\theta}\cos\theta)
$$

If we insert this into the Lagrangian, we get

$$
\mathcal{L} = \frac{1}{2}m(\frac{1}{2}(\frac{\ell}{m} -d\dot{\theta}\cos\theta)^2 +d^2\dot{\theta}^2 + \dot{\theta}(\frac{\ell}{m} -d\dot{\theta}\cos\theta)d\cos\theta) +mgd\cos\theta
$$

$$
= \frac{1}{2}m(\frac{\ell^2}{2m^2} + \frac{d^2\dot{\theta}^2}{2}\cos^2\theta + d^2\dot{\theta}) + mgd\sin\theta
$$

\subsection*{c)}
We can now find the e.o.m

$$
\frac{\partial \mathcal{L}}{\partial \theta} = -mgd\sin\theta + \frac{1}{2}md^2\dot{\theta}^2\cos\theta\sin\theta
$$

$$
\frac{\partial \mathcal{L}}{\partial \dot{\theta}} = -\frac{1}{2}md^2\dot{\theta}\cos^2\theta +md^2\dot{\theta} = md^2(\dot{\theta} - \frac{1}{2}\dot{\theta}\cos^2\theta)
$$
$$
\frac{d}{dt}\left(\frac{\partial \mathcal{L}}{\partial \dot{\theta}}\right) = md^2(\ddot{\theta} - \frac{1}{2}\ddot{\theta}\cos^2\theta + \dot{\theta}^2\sin\theta\cos\theta)
$$

This gives us that
$$
md^2(\ddot{\theta} - \frac{1}{2}\ddot{\theta}\cos^2\theta + \dot{\theta}^2\sin\theta\cos\theta) +mgd\sin\theta - \frac{1}{2}md^2\dot{\theta}^2\cos\theta\sin\theta = 0
$$
$$
\Rightarrow \ddot{\theta}(1-\frac{1}{2}\cos^2\theta) + \frac{\dot{\theta}^2}{2}\sin\theta\cos\theta + \frac{g}{d}\sin\theta = 0
$$

\subsection*{d)}
We can find a small-angle solution to this problem. If we Taylor expand the above expression around $\theta = 0$. We get that the first order Taylor series is

$$
T_2(f(\theta)) = \frac{1}{2}\ddot{\theta}+\frac{g}{d}\theta + \frac{1}{2}\theta\dot{\theta}^2 = 0
$$

We are only interested in the first order, and for small $\theta$ see get that $\dot{\theta}^2 \rightarrow 0$, so this expression becomes

$$
\frac{1}{2}\ddot{\theta}+\frac{g}{d}\theta= 0
$$

This is a harmonic oscillator equation

$$
\ddot{\theta} = - \frac{2g}{d}\theta
$$

The angular frequency of this system will be

$$
\omega = \sqrt{\frac{2g}{d}}
$$



\end{document}


